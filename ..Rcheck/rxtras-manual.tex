\nonstopmode{}
\documentclass[letterpaper]{book}
\usepackage[times,inconsolata,hyper]{Rd}
\usepackage{makeidx}
\makeatletter\@ifl@t@r\fmtversion{2018/04/01}{}{\usepackage[utf8]{inputenc}}\makeatother
% \usepackage{graphicx} % @USE GRAPHICX@
\makeindex{}
\begin{document}
\chapter*{}
\begin{center}
{\textbf{\huge Package `rxtras'}}
\par\bigskip{\large \today}
\end{center}
\ifthenelse{\boolean{Rd@use@hyper}}{\hypersetup{pdftitle = {rxtras: Extra functionality}}}{}
\begin{description}
\raggedright{}
\item[Type]\AsIs{Package}
\item[Title]\AsIs{Extra functionality}
\item[Version]\AsIs{0.1.4}
\item[Author]\AsIs{person(``Kornel'', ``Labun'', email = ``kornellabun@gmail.com'',
role = c(``aut'', ``cre''))}
\item[Maintainer]\AsIs{}\email{kornellabun@gmail.com}\AsIs{}
\item[Description]\AsIs{Contains functions that are extension to basic R,
some of them add functionality to different packages.
This package consists of variety of random functions packed for convienience.}
\item[License]\AsIs{MIT + file LICENSE}
\item[LazyData]\AsIs{TRUE}
\item[BugReports]\AsIs{}\url{https://github.com/JokingHero/rxtras/issues}\AsIs{}
\item[URL]\AsIs{}\url{https://github.com/JokingHero/rxtras}\AsIs{}
\item[RoxygenNote]\AsIs{7.3.3}
\item[Encoding]\AsIs{UTF-8}
\item[Depends]\AsIs{R (>= 4.1.0), methods, ggplot2 (>= 3.4.0)}
\item[Suggests]\AsIs{knitr, rmarkdown, testthat, htmltools, IRanges}
\item[VignetteBuilder]\AsIs{knitr}
\item[SystemRequirements]\AsIs{pandoc (>= 2.0) - http://pandoc.org (for building
vignettes)}
\end{description}
\Rdcontents{Contents}
\HeaderA{as.pngpdf}{Wrapper around gglplot2::ggsave with presets.}{as.pngpdf}
%
\begin{Description}
Wrapper around gglplot2::ggsave with presets.
\end{Description}
%
\begin{Usage}
\begin{verbatim}
as.pngpdf(
  filename,
  plot = ggplot2::last_plot(),
  scale = 1,
  width = 12,
  height = 8,
  units = "in",
  dpi = 400
)
\end{verbatim}
\end{Usage}
%
\begin{Arguments}
\begin{ldescription}
\item[\code{filename}] File name to create on disk.

\item[\code{plot}] Plot to save, defaults to last plot displayed.

\item[\code{scale}] Multiplicative scaling factor.

\item[\code{width}, \code{height}] Plot size in units expressed by the \code{units} argument.
If not supplied, uses the size of the current graphics device.

\item[\code{units}] One of the following units in which the \code{width} and \code{height}
arguments are expressed: \code{"in"}, \code{"cm"}, \code{"mm"} or \code{"px"}.

\item[\code{dpi}] Plot resolution. Also accepts a string input: "retina" (320),
"print" (300), or "screen" (72). Only applies when converting pixel units,
as is typical for raster output types.
\end{ldescription}
\end{Arguments}
%
\begin{Value}
A plot saved as .png and .pdf, returns nothing.
\end{Value}
\HeaderA{clipboard}{Copy an object in the clipboard}{clipboard}
%
\begin{Description}
Copy an object in the clipboard (needs xclip to be installed)
\end{Description}
%
\begin{Usage}
\begin{verbatim}
clipboard(x, sep = "\t", row.names = FALSE, col.names = FALSE)
\end{verbatim}
\end{Usage}
%
\begin{Arguments}
\begin{ldescription}
\item[\code{x}] The object to copy.

\item[\code{sep}] A character to be used as separator for each column of the object

\item[\code{row.names}] Copy row names (default is FALSE)

\item[\code{col.names}] Copy column names (default is TRUE)
\end{ldescription}
\end{Arguments}
%
\begin{Value}
copy the object as character in the clipboard
\end{Value}
%
\begin{Author}
freecube source:http://stackoverflow.com/questions/10959521/how-to-write-to-clipboard-on-ubuntu-linux-in-r
\end{Author}
\HeaderA{cpaste}{Paste an object from the clipboard}{cpaste}
%
\begin{Description}
Paste an object from the clipboard (needs xsel to be installed)
\end{Description}
%
\begin{Usage}
\begin{verbatim}
cpaste(header = FALSE, ...)
\end{verbatim}
\end{Usage}
%
\begin{Arguments}
\begin{ldescription}
\item[\code{header}] Is there a header?

\item[\code{...}] What arguments to pass to the read.delim function?
\end{ldescription}
\end{Arguments}
%
\begin{Value}
copy the object as character in the clipboard
\end{Value}
%
\begin{Author}
isomorphismes source:http://stackoverflow.com/questions/13438556/how-do-i-copy-and-paste-data-into-r-from-the-clipboard
\end{Author}
\HeaderA{get\_palette\_from\_theme}{Get Palette from Current Theme}{get.Rul.palette.Rul.from.Rul.theme}
%
\begin{Description}
Reads the palette setting from the current ggplot2 theme.
Returns "adj\_colorblind" as default if no palette is set.
\end{Description}
%
\begin{Usage}
\begin{verbatim}
get_palette_from_theme()
\end{verbatim}
\end{Usage}
%
\begin{Value}
Character string with palette name
\end{Value}
\HeaderA{ggplot2\_settings\_on}{Set options for ggplot2}{ggplot2.Rul.settings.Rul.on}
%
\begin{Description}
Set theme to theme science and colors and fill.
\end{Description}
%
\begin{Usage}
\begin{verbatim}
ggplot2_settings_on()
\end{verbatim}
\end{Usage}
\HeaderA{plotRanges}{Plot genomic ranges from the IRanges object or GRanges object.}{plotRanges}
%
\begin{Description}
Plot genomic ranges from the IRanges object or GRanges object.
\end{Description}
%
\begin{Usage}
\begin{verbatim}
plotRanges(
  x,
  xlim = x,
  main = deparse(substitute(x)),
  col = "black",
  sep = 0.5,
  ...
)
\end{verbatim}
\end{Usage}
%
\begin{Arguments}
\begin{ldescription}
\item[\code{x}] a IRanges object.

\item[\code{xlim}] like in plot.

\item[\code{main}] A title of the plot.

\item[\code{col}] color. Default "black". e.g. "blue".

\item[\code{sep}] Default 0.5.

\item[\code{...}] Other params you can pass to plot function.
\end{ldescription}
\end{Arguments}
%
\begin{Value}
A plot.
\end{Value}
%
\begin{Examples}
\begin{ExampleCode}
library(IRanges)
plotRanges(IRanges(1:10, width=10:1, names=letters[1:10]))
\end{ExampleCode}
\end{Examples}
\HeaderA{rxtras}{rxtras package}{rxtras}
\aliasA{rxtras-package}{rxtras}{rxtras.Rdash.package}
%
\begin{Description}
Extra functions and addins that you might find useful.
\end{Description}
%
\begin{SeeAlso}
Useful links:
\begin{itemize}

\item{} \url{https://github.com/JokingHero/rxtras}
\item{} Report bugs at \url{https://github.com/JokingHero/rxtras/issues}

\end{itemize}


\end{SeeAlso}
\HeaderA{scale\_colour\_science}{Science Color Scale for Discrete Data}{scale.Rul.colour.Rul.science}
%
\begin{Description}
Discrete color scale using science palette colors.
Automatically uses the palette from the current theme, unless explicitly overridden.
\end{Description}
%
\begin{Usage}
\begin{verbatim}
scale_colour_science(palette = NULL, ...)
\end{verbatim}
\end{Usage}
%
\begin{Arguments}
\begin{ldescription}
\item[\code{palette}] Palette name: "adj\_colorblind", "pastel", or "resurrect".
If NULL, reads from current theme.

\item[\code{...}] Additional arguments passed to ggplot2::scale\_colour\_manual()
\end{ldescription}
\end{Arguments}
%
\begin{Value}
A discrete color scale
\end{Value}
\HeaderA{scale\_colour\_science\_div}{Science Diverging Color Scale (3-color)}{scale.Rul.colour.Rul.science.Rul.div}
%
\begin{Description}
Diverging color scale using blue-white-red gradient derived from
the science palette colors. Automatically uses the palette from the current
theme, unless explicitly overridden.
\end{Description}
%
\begin{Usage}
\begin{verbatim}
scale_colour_science_div(palette = NULL, ...)
\end{verbatim}
\end{Usage}
%
\begin{Arguments}
\begin{ldescription}
\item[\code{palette}] Palette name: "adj\_colorblind", "pastel", or "resurrect".
If NULL, reads from current theme.

\item[\code{...}] Additional arguments passed to ggplot2::scale\_colour\_gradient2()
\end{ldescription}
\end{Arguments}
%
\begin{Value}
A diverging color scale
\end{Value}
\HeaderA{scale\_colour\_science\_seq}{Science Sequential Color Scale (2-color)}{scale.Rul.colour.Rul.science.Rul.seq}
%
\begin{Description}
Sequential color scale using blue-red gradient derived from
the science palette colors. Automatically uses the palette from the current
theme, unless explicitly overridden.
\end{Description}
%
\begin{Usage}
\begin{verbatim}
scale_colour_science_seq(palette = NULL, ...)
\end{verbatim}
\end{Usage}
%
\begin{Arguments}
\begin{ldescription}
\item[\code{palette}] Palette name: "adj\_colorblind", "pastel", or "resurrect".
If NULL, reads from current theme.

\item[\code{...}] Additional arguments passed to ggplot2::scale\_colour\_gradient()
\end{ldescription}
\end{Arguments}
%
\begin{Value}
A sequential color scale
\end{Value}
\HeaderA{scale\_fill\_science}{Science Fill Scale for Discrete Data}{scale.Rul.fill.Rul.science}
%
\begin{Description}
Discrete fill scale using science palette colors.
Automatically uses the palette from the current theme, unless explicitly overridden.
\end{Description}
%
\begin{Usage}
\begin{verbatim}
scale_fill_science(palette = NULL, ...)
\end{verbatim}
\end{Usage}
%
\begin{Arguments}
\begin{ldescription}
\item[\code{palette}] Palette name: "adj\_colorblind", "pastel", or "resurrect".
If NULL, reads from current theme.

\item[\code{...}] Additional arguments passed to ggplot2::scale\_fill\_manual()
\end{ldescription}
\end{Arguments}
%
\begin{Value}
A discrete fill scale
\end{Value}
\HeaderA{scale\_fill\_science\_div}{Science Diverging Fill Scale (3-color)}{scale.Rul.fill.Rul.science.Rul.div}
%
\begin{Description}
Diverging fill scale using blue-white-red gradient derived from
the science palette colors. Automatically uses the palette from the current
theme, unless explicitly overridden.
\end{Description}
%
\begin{Usage}
\begin{verbatim}
scale_fill_science_div(palette = NULL, ...)
\end{verbatim}
\end{Usage}
%
\begin{Arguments}
\begin{ldescription}
\item[\code{palette}] Palette name: "adj\_colorblind", "pastel", or "resurrect".
If NULL, reads from current theme.

\item[\code{...}] Additional arguments passed to ggplot2::scale\_fill\_gradient2()
\end{ldescription}
\end{Arguments}
%
\begin{Value}
A diverging fill scale
\end{Value}
\HeaderA{scale\_fill\_science\_seq}{Science Sequential Fill Scale (2-color)}{scale.Rul.fill.Rul.science.Rul.seq}
%
\begin{Description}
Sequential fill scale using blue-red gradient derived from
the science palette colors. Automatically uses the palette from the current
theme, unless explicitly overridden.
\end{Description}
%
\begin{Usage}
\begin{verbatim}
scale_fill_science_seq(palette = NULL, ...)
\end{verbatim}
\end{Usage}
%
\begin{Arguments}
\begin{ldescription}
\item[\code{palette}] Palette name: "adj\_colorblind", "pastel", or "resurrect".
If NULL, reads from current theme.

\item[\code{...}] Additional arguments passed to ggplot2::scale\_fill\_gradient()
\end{ldescription}
\end{Arguments}
%
\begin{Value}
A sequential fill scale
\end{Value}
\HeaderA{scale\_gradient}{Scale gradient for ggplot2}{scale.Rul.gradient}
%
\begin{Description}
Depending on min and max will apply gradient to the plot
if min < 0, use 3 colours and set limits to -max\_val, +max\_val
else use 2 colours and set limits to 0, max

Alternative dark red-blue: c("\#750e13", "\#ffffff", "\#003a6d")
Alternative violet-green: c("\#491d8b", "\#ffffff", "\#004144")
\end{Description}
%
\begin{Usage}
\begin{verbatim}
scale_gradient(
  min,
  max,
  aesthetics = "fill",
  gradient = c("#D62929", "#ffffff", "#2982D6"),
  na.value = "#dddddd"
)
\end{verbatim}
\end{Usage}
%
\begin{Arguments}
\begin{ldescription}
\item[\code{min}] Minimal value in the data

\item[\code{max}] Maximal value in the data

\item[\code{aesthetics}] "colour" or "fill" or both

\item[\code{gradient}] A vector of 3 colours: low, middle and high

\item[\code{na.value}] What colour to put on the na.value
\end{ldescription}
\end{Arguments}
%
\begin{Value}
a scale\_fill\_gradient
\end{Value}
\HeaderA{science\_palette}{Get Science Palette Colors}{science.Rul.palette}
%
\begin{Description}
Returns colors from the science palettes. The function
respects the palette setting from the current theme if no palette is
specified.
\end{Description}
%
\begin{Usage}
\begin{verbatim}
science_palette(n, palette = NULL)
\end{verbatim}
\end{Usage}
%
\begin{Arguments}
\begin{ldescription}
\item[\code{n}] Number of colors to return

\item[\code{palette}] Palette name: "adj\_colorblind" (default), "pastel", or "resurrect"
\end{ldescription}
\end{Arguments}
%
\begin{Value}
Character vector of hex color codes
\end{Value}
\HeaderA{substrRight}{Extract substrings in a character vector counting from right.}{substrRight}
%
\begin{Description}
Extract substrings in a character vector counting from right.
\end{Description}
%
\begin{Usage}
\begin{verbatim}
substrRight(x, n)
\end{verbatim}
\end{Usage}
%
\begin{Arguments}
\begin{ldescription}
\item[\code{x}] a character vector.

\item[\code{n}] integer. The first element to be replaced.
\end{ldescription}
\end{Arguments}
%
\begin{Value}
A substracted character from \code{x}.
\end{Value}
%
\begin{Examples}
\begin{ExampleCode}
substrRight("ABCD", 2)
#CD

\end{ExampleCode}
\end{Examples}
\HeaderA{theme\_science}{Set theme for ggplot2}{theme.Rul.science}
%
\begin{Description}
The theme set supports 10 colors in the colorblind friendly, yet
beautiful pastille colors, and afterwards it backs out to 64 colors from
resurrect palette. Theme is based on `ggthemes::theme\_pander()`.
Colors are automatically applied to discrete scales. Continuous scales
can use the palette-specific gradient functions.
\end{Description}
%
\begin{Usage}
\begin{verbatim}
theme_science(
  base_size = 16,
  base_family = "sans",
  nomargin = FALSE,
  fc = "black",
  gM = TRUE,
  gm = TRUE,
  gc = "grey",
  gl = "dashed",
  boxes = TRUE,
  bc = "white",
  pc = "transparent",
  lp = "top",
  axis = 1,
  palette = "adj_colorblind"
)
\end{verbatim}
\end{Usage}
%
\begin{Arguments}
\begin{ldescription}
\item[\code{base\_size}] Font size

\item[\code{base\_family}] Font family

\item[\code{nomargin}] If true will remove all margins.

\item[\code{fc}] text colour, default is black

\item[\code{gM}] grid Major

\item[\code{gm}] grid minor

\item[\code{gc}] grid color

\item[\code{gl}] grid linetype

\item[\code{boxes}] whether to keep boxes around the plot

\item[\code{bc}] background color, default white

\item[\code{pc}] panel colour background, default transparent

\item[\code{lp}] legend position, default is top

\item[\code{axis}] 0, 1, 2, 3 - default is 1

\item[\code{palette}] Palette name: "adj\_colorblind" (default), "pastel", or "resurrect"
\end{ldescription}
\end{Arguments}
%
\begin{Value}
a ggplot2 theme
\end{Value}
%
\begin{Author}
See ggtheme package
\end{Author}
\HeaderA{validate\_palette}{Validate Palette Parameter}{validate.Rul.palette}
%
\begin{Description}
Validates that the palette parameter is one of the allowed values.
Throws an error with helpful message if invalid.
\end{Description}
%
\begin{Usage}
\begin{verbatim}
validate_palette(palette)
\end{verbatim}
\end{Usage}
%
\begin{Arguments}
\begin{ldescription}
\item[\code{palette}] Character string with palette name
\end{ldescription}
\end{Arguments}
%
\begin{Value}
The validated palette name
\end{Value}
\printindex{}
\end{document}
